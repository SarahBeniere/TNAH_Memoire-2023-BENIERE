% Introdction

\begin{quotation}
    \enquote{Le monde semble être divisé en deux parties~: les endroits où les Juifs ne peuvent pas vivre et ceux où ils ne peuvent pas entrer\footcite[p.~27, citation de Haïm Weizmann]{Bensoussan2020}.}
\end{quotation}

Le projet EHRI\footnote{Site de l'EHRI~: \texttt{\href{https://www.ehri-project.eu/}{https://www.ehri-project.eu/}}.} a été lancé en 2010 grâce au financement du septième programme-cadre de l'Union Européenne pour la période 2007-2013 (FP7)\footnote{Évaluation de la Commission européenne du 25/01/2016 sur le septième programme-cadre~: \texttt{\href{https://ec.europa.eu/commission/presscorner/detail/en/MEMO_16_146}{https://ec.europa.eu/commission/presscorner/detail/en/MEMO\_16\_146}}.} dédié à la recherche et à l'innovation. Depuis 2014, le projet bénéficie du financement des programmes-cadres Horizon 2020 (2014-2020)\footnote{Présentation du programme Horizon 2020 sur le site du Ministère de l'Enseignement Supérieur et de la Recherche~: \texttt{\href{https://www.enseignementsup-recherche.gouv.fr/fr/horizon-2020-le-programme-de-l-union-europeenne-pour-la-recherche-et-l-innovation-46458}{https://www.enseignementsup-recherche.gouv.fr/fr/horizon-2020}}.} et Horizon Europe (2021-2027).\footnote{Présentation du programme Horizon Europe~: \texttt{\href{https://www.horizon-europe.gouv.fr/presentation-du-programme-horizon-europe-24104}{https://www.horizon-europe.gouv.fr/}}.} En 2018, l'EHRI est devenue une organisation permanente et souhaite acquérir le statut d'ERIC\footnote{Pour plus d'informations~: \texttt{\href{https://research-and-innovation.ec.europa.eu/strategy/strategy-2020-2024/our-digital-future/european-research-infrastructures/eric_en}{https://research-and-innovation.ec.europa.eu/strategy/}}.} d'ici à janvier 2025, commémorant ainsi le 80\ieme{} anniversaire de la libération d'Auschwitz.

L'EHRI est une organisation transnationale comptant actuellement vingt-sept partenaires à travers l'Europe, Israël et les États-Unis. L'infrastructure est coordonnée par le \textit{NIOD Institute for War, Holocaust and Genocides Studies}\footnote{Site du NIOD~: \texttt{\href{https://www.niod.nl/en}{https://www.niod.nl/en}}.}, basé à Amsterdam aux Pays-Bas, et regroupe des centres d'archives, des bibliothèques, des musées et des instituts de recherche. En tant qu'infrastructure de recherche, la principale mission de l'EHRI est de promouvoir la recherche sur la Shoah\footnote{Pour une définition du terme \enquote{Shoah} et l'utilisation que nous en faisons, voir notre glossaire en Annexe \ref{Glossaire}.}. En effet, après la Seconde Guerre mondiale, les sources\footnote{Sur l'usage du terme \enquote{source}, l'EHRI rassemble des documents que nous considérons comme des sources dites \enquote{primaires}, constituant l'objet d'étude des historien$\cdot$ne$\cdot$s. En tant qu'ingénieur$\cdot$e, et dans un souci de neutralité, nous lui préférerons le terme \enquote{document}, qui nous semble plus adapté.} concernant son histoire se sont dispersées et certaines sont devenues difficiles d'accès. L'EHRI a donc créé un portail\footnote{Portail de l'EHRI~: \texttt{\href{https://portal.ehri-project.eu}{https://portal.ehri-project.eu}}.} permettant d'accéder aux informations concernant la conservation des documents liés à la Shoah.  

Depuis mai 2015, dans le cadre du programme Horizon 2020, l'équipe-projet ALMAnaCH\footnote{Projets de l'équipe ALMAnaCH~: \texttt{\href{https://almanach.inria.fr/projects-fr.html}{https://almanach.inria.fr/projects-fr.html}}.} d'Inria, apporte son soutien à l'EHRI. Coordonnés par Laurent Romary, les membres de l'équipe affiliés au projet EHRI s'attachent à développer des méthodes et des outils informatiques permettant de faciliter la recherche archivistique sur la Shoah.  

Le présent mémoire rend compte de notre expérience au sein de l'équipe ALMAnaCH sur le projet EHRI. Nous nous intéresserons à la question de l'édition scientifique numérique, en nous appuyant sur les éditions en ligne de l'EHRI. Notre réflexion s'articule autour de trois axes, constituant les grandes parties de ce travail. Nous nous attacherons dans un premier temps à définir l'édition en tant que concept et la chaîne éditoriale. Nous présenterons également notre corpus de travail et son positionnement d'un point de vue diplomatique. Dans un second temps, nous nous pencherons sur les étapes et la réflexion nécessaires à la création d'un modèle d'encodage. Enfin, nous nous intéresserons à la publication des éditions numériques, et plus particulièrement à l'outil TEI Publisher.