% Chronologie de l'histoire de la Shoah

Cette chronologie non exhaustive a été établie à partir des ressources suivantes~:
\begin{itemize}
    \item \cite{Bensoussan2020}.
    \item \cite{Bruttmann2020}.
    \item \cite{Memorial}.
\end{itemize}

\bigskip
\bigskip

\begin{description}

    \item[XIII\ieme{} siècle] Création des premiers ghettos et premières grandes migrations de l'Europe occidentale vers l'Europe de l'Est, en particulier vers la Pologne et la Lituanie.
    
    \item[1215] Le IV\ieme{} concile du Latran impose aux Juifs le port d'un signe distinctif~: la \enquote{rouelle} (petit pièce de tissu jaune).
    
    \item[15 novembre 1879] Heinrich Gothard von Treitschke publie un article antisémite dans les \textit{Annales prussiennes}, dont les Nazis reprendront la formule~: \enquote{Les Juifs sont notre malheur~!}.
    
    \item[1881-1882] En Russie, les \textit{pogroms} accélèrent l'émigration des Juifs de Russie.
    
    \item[Années 1900] Montée du darwinisme social et de l'idéologie eugéniste en Allemagne.

    \item[5 janvier 1919] Création du Parti ouvrier allemand, qui devient le Parti national-socialiste des travailleurs allemands (NSDAP) en 1921.
    
    \item[Années 1920] Montée de la xénophobie et de l'antisémitisme. La défaite de l'Allemagne à l'issue de la Première Guerre mondiale pousse la classe moyenne allemande vers le nationalisme d'extrême-droite.
    
    \item[1919-1921] Nombreux \textit{pogroms} en Pologne.
    
    \item[24 février 1920] Premier programme politique du NSDAP. Il entend retirer aux Juifs la citoyenneté allemande et réduire leurs droits à ceux des étrangers. Un point révoque l'accès aux emplois de la fonction publique pour les Juifs, et un autre souhaite expulser les Juifs entrés en Allemagne après le 2 août 1914.
    
    \item[18 juillet 1925] Publication de \textit{Mein Kampf} (Adolf Hitler).
    
    \item[30 janvier 1933] Adolf Hitler est appelé à la Chancellerie.
    
    \item[5 mars 1933] Élections législatives anticipées. Le NSDAP obtient 44~\%{} des suffrages.
    
    \item[27 février 1933] Incendie du Reichstag. 
    
    \item[22 mars 1933] Ouverture du premier camp de concentration à Dachau (Allemagne).
    
    \item[23 mars 1933] Hitler obtient les pleins pouvoirs.

    \item[Avril 1933] Les fonctionnaires juifs sont révoqués et les avocats juifs sont radiés du barreau.
    
    \item[26 avril 1933] Création de la Gestapo.

    \item[14 juillet 1933] Loi sur la stérilisation forcée. Les personnes atteintes de \enquote{maladies héréditaires} (physiques ou mentales) et les criminels \enquote{irrécupérables et dangereux} sont stérilisés avec l'appui des psychiatres.

    \item[14 juillet 1933] Le NSDAP devient le seul parti autorisé en Allemagne.

    \item[25 août 1933] L'Accord de \enquote{\textit{haavara}} (de l'hébreu \enquote{transfert}) permet de négocier le départ de 60~000 Juifs vers la Palestine.

    \item[29 juin 1934] Nuit des longs couteaux.
    
    \item[Juillet 1934] Création de l'Inspection des camps de concentration.

    \item[1\ier{} août 1934] Hitler se proclame \enquote{Führer} et Chancelier du Reich.
    
    \item[15 septembre 1935] Lois de Nuremberg. Séparation physique des Juifs des autres Allemands.

    \item[7 mars 1936] Entrée de la Wehrmacht en Rhénanie (zone démilitarisée par le Traité de Versailles en 1919).

    \item[12 juillet 1936] Ouverture du camp de concentration d'Oranienburg-Sachsenhausen.

    \item[15 juillet 1937] Ouverture du camp de concentration de Buchenwald.
    
    \item[Mars 1938] Aux États-Unis, 82\%{} de la population s'oppose à l'accueil de réfugiés juifs venus d'Allemagne et d'Autriche.
    
    \item[12--13 mars 1938] Annexion de l'Autriche (\textit{Anschlu\ss{}})

    \item[Avril 1938] Création d'un centre d'émigration juive à Vienne, dirigé par Adolf Eichmann. En six mois, le quart de la communauté juive est expulsée.
    
    \item[26 avril 1938] Les Juifs doivent déclarer tous leurs biens.
    
    \item[Mai 1938] Première législation antisémite en Hongrie.
    
    \item[14 juin 1938] Début de l'aryanisation forcée.

    \item[Juillet 1938] Les médecins juifs doivent demander l'autorisation d'exercer et limiter leur pratique à une patientèle juive.
    
    \item[6--15 juillet 1938] La conférence d'Évian réunit trente-deux états pour répondre à la question des réfugiés juifs. La Hongrie, la Roumanie et la Pologne envoient des \enquote{observateurs}. \enquote{Nul ne conteste à l'Allemagne sa souveraineté à l'égard de ses nationaux}.

    \item[Août 1938] Les Juifs doivent ajouter un prénom de marquage à leur prénom courant~: \enquote{Israël} pour les hommes, \enquote{Sara} pour les femmes.
    
    \item[8 août 1938] Ouverture du camp de concentration de Mauthausen (Autriche).

    \item[30 septembre 1938] Accords de Munich.
    
    \item[9--10 novembre 1938] Nuit de Cristal (\textit{Kristallnacht}).    
    
    \item[Décembre 1938] En Autriche, les Juifs perdent leur permis de conduire.

    \item[3 décembre 1938] Les propriétaires juifs reçoivent l'ordre de vendre tous leurs biens restants.
    
    \item[31 décembre 1938] Les travailleurs indépendants juifs doivent cesser toute activité.

    \item[Janvier 1939] Création du \enquote{Bureau central pour l'émigration des Juifs} à Berlin, dirigé par Reinhard Heydrich.
    
    \item[Mars 1939] La Suisse ferme ses frontières aux réfugiés juifs.

    \item[15 mars 1939] Invasion de la Tchécoslovaquie et violation des Accords de Munich.

    \item[Avril 1939] En Autriche, les locataires juifs perdent tous leurs droits face à leurs propriétaires.
    
    \item[Mai 1939] Recensement général de la population allemande. Les fiches des Juifs sont marquées de la lettre \enquote{J}.
    \item[Mai 1939] Ouverture du camp de concentration de  Ravensbrück.
    
    \item[Septembre 1939] Couvre-feu à 20h pour les Juifs uniquement.

    \item[1\ier{} sepembre 1939] Invasion de la Pologne par l'Allemagne.

    \item[25 septembre 1939] La Pologne est divisée en quatre zones réparties entre l'Allemagne, l'URSS et la Lituanie.
    
    \item[27 septembre 1939] Début de l'\enquote{Aktion T4} à l'asile de Kocborowo (Pologne).

    \item[Octobre 1939] Début des déportations vers Nisko (Pologne).
    
    \item[8 octobre 1939] Création du ghetto de Piotrkow.
    
    \item[9 octobre 1939] Début du recensement des malades mentaux.
    
    \item[26 octobre 1939] Instauration du travail forcé pour les personnes âgées de 14 à 60 ans dans le \enquote{gouvernement général} (zone de Pologne occupée par l'Allemagne).

    \item[28 novembre 1939] Créations des \enquote{Conseils juifs} dans le \enquote{gouvernement général}. La communauté juive prend elle-même en charge les tâches administratives liées au recensement, à la spoliation et à la déportation de sa population.
    
    \item[Décembre 1939] Exclusion des Juifs allemands des distributions spéciales de nourritures.

    \item[11 décembre 1939] Tout changement de domicile est interdit aux Juifs polonais.

    \item[Mars 1940] Identification des Juifs par un \enquote{J} sur leur carte alimentaire.

    \item[20 avril 1940] Création du ghetto de Lodz.
    
    \item[9 avril 1940] Invasion du Danemark et du sud de la Norvège.

    \item[27 avril 1940] Ouverture du camp de concentration d'Auschwitz.

    \item[2 octobre 1940] Création du ghetto de Varsovie.

    \item[Janvier 1941] Création des \enquote{\textit{Einsatzgruppen}}.
    
    \item[3 mars 1941] Création du ghetto de Cracovie.

    \item[24 mars 1941] Création du ghetto de Lublin.
    
    \item[Avril 1941] Début de l'\enquote{Aktion 14f13}. Mise à mort des détenus incapables de travailler dans les camps de concentration.

    \item[29 juin 1941] Opération Barbarossa (invasion de la Russie).

    \item[15 août 1941] Heinrich Himmler se rend à Minsk et décide d'expérimenter le meurtre à l'aide des camions à gaz, déjà utilisés pour l'assassinat des malades mentaux.

    \item[24 août 1941] Hitler annonce la fin de l'\enquote{Aktion T4}.
    
    \item[Septembre 1941] Les Juifs allemands de plus de 6 ans doivent porter l'étoile de David sur le côté gauche de leur vêtement. Ils ne peuvent plus quitter leur commune de résidence.

    \item[5--6 septembre 1941] Les prisonniers de guerre soviétiques sont assassinés par le gaz Zyklon B à Auschwitz.

    \item[16 et 18 septembre 1941] Premiers essais des camions à gaz à Mohilev et à Minsk (Biélorussie).

    \item[29 septembre 1941] Massacre de Babi Yar (Ukraine).

    \item[Octobre 1941] Le ghetto de Lodz devint un centre de transit pour les Juifs déportés du Reich.

    \item[1\ier{} octobre 1941] Début des travaux visant à transformer le Birkenau (Auschwitz) en camp d'extermination.

    \item[23 octobre 1941] L'Europe allemande est définitivement fermée à l'émigration juive.

    \item[1\ier{} novembre 1941] Construction du centre de mise à mort de Belzec (Pologne).

    \item[3 novembre 1941] Essais des camions à gaz à Sachsenhausen.

    \item[8 novembre 1941] Création du ghetto de Lwow (Ukraine).

    \item[24 novembre 1941] Ouverture du camp de concentration de Terezin (République Tchèque).

    \item[7 décembre 1941] Attaque de Pearl Harbor.

    \item[8 décembre 1941] Ouverture du camp d'extermination de Chelmno, près de Lodz. Début des exterminations par le gaz.
    
    \item[20 janvier 1942] Conférence de Wansee. Début de la \enquote{Solution finale}.

    \item[17 ars 1942] Ouverture du camp d'extermination de Belzec (Pologne).

    \item[7 mai 1942] Ouverture du camp d'extermination de Sobibor (Pologne).

    \item[22 juillet 1942] Ouverture du camp d'extermination de Treblinka (Pologne) et déportation des Juifs du ghetto de Varsovie.

    \item[13--14 mars 1943] Liquidation du ghetto de Cracovie.

    \item[19 avril 1943 -- 16 mai 1943] Révolte du ghetto de Varsovie.

    \item[21 juin 1943] Décision de liquider tous les ghettos de Pologne.

    \item[3 novembre 1943] Début de l'Opération \enquote{\textit{Erntefest}} (\enquote{Fête de la Moisson}~: extermination Juifs de Lublin (environ 45000).

    \item[19 mars 1944] Invasion de la Hongrie.

    \item[Août 1944] Liquidation du ghetto de Lodz et déportation des survivants à Auschwitz.

    \item[30 octobre 1944] Déportation des Juifs de Theresienstadt vers Auschwitz.

    \item[27 janvier 1945] Libération d'Auschwitz.

    \item[30 avril 1945] Suicide d'Adolf Hitler.

    \item[7 mai 1945] Capitulation de l'Allemagne.
    
    \item[18 octobre 1945 -- 1\ier{} octobre 1946] Procès de Nuremberg.

\end{description}