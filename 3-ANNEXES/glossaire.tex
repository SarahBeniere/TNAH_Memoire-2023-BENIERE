% Glossaire

\label{Glossaire}

Nous avons rédigé les définitions de ce glossaire à l'aide des références suivantes~:
\begin{itemize}
    \item \cite{Bensoussan2020}.
    \item \cite{Bruttmann2020}.
\end{itemize}

\bigskip
\bigskip

\begin{description}
    
    \item[Antisémitisme] Idéologie selon laquelle \enquote{le Juif} serait responsable de tous les maux. L'antisémitisme prend la forme de discrimination, voire de violences, à l'encontre des Juif$\cdot$ve$\cdot$s en raison de leur ethnie. 
    
    \item[Aryanisation] Processus visant à transférer la propriété d'une personne juive à un \enquote{aryen}. Le terme renvoie à une \enquote{opération de décontamination}. Il y a eu deux phases à l'aryanisation~: l'\enquote{aryanisation volontaire} et l'\enquote{aryanisation forcée}. Cette deuxième phase commence le 14 juin 1938 lorsque le ministre de l'Intérieur, Whilhelm Frick, décide de transférer les biens et propriétés juifs à l'État pour les revendre à crédit aux classes moyennes allemandes.      
    
    \item[Camp de concentration] (\textit{Konzentrationslager}). Symbole de la répression nazie, le camp de concentration est à l'origine un camp d'internement des opposants au régime. Le premier camp de concentration nazi est celui de Dachau en Allemagne.
    
    \item[Camp d'extermination] (Centre de mise à mort). Après les expérimentations de la mort par le gaz menés sur les malades mentaux lors de l'\enquote{Aktion T4}, la méthode est utilisée dans les camps pour exterminer le plus de victimes en un minimum de temps. Le premier camp d'extermination est celui de Chelmno, ouvert le 8 décembre 1941.
    
    \item[Déportation] Déplacement d'une personne ou d'un groupe de personne contre son gré. Durant la Shoah, les Juif$\cdot$ve$\cdot$s étaient déporté$\cdot$e$\cdot$s dans les ghettos, puis dans les camps de concentration et les camps d'extermination.
    
    \item[Groupes d'action spéciale] (\textit{Einsatzgruppen}). Unités mobiles du Reich allemand, chargées d'assassiner les Juif$\cdot$ve$\cdot$s et les commissaires politiques communistes. Au début, les groupes sont composés de presque 3~000 hommes, tous volontaires.
    
    \item[Émigration] Action volontaire de quitter son lieu de résidence ou son pays. L'objectif premier du régime nazi était de forcer les populations juives à émigrer hors d'Allemagne par le biais d'intimidations. Cette politique s'est soldée par un échec certain avec l'expansion du Reich, menant ensuite à la politique d'extermination de la population juive d'Europe.
        
    \item[Espace vital] (\textit{Lebensraum}). Concept du pangermanisme selon lequel le peuple allemand (comprendre les \enquote{aryens}) doit disposer d'un espace suffisant pour vivre. L'extension du territoire est vue comme une nécessité, ce qui permet au régime nazi de justifier ses invasions successives.
    
    \item[Génocide] Destruction partielle ou totale intentionnelle d'une population en raison de son ethnie ou de sa religion.

    \item[Gestapo] Abréviation de \enquote{\textit{Geheime Staatspolizei}}. Police politique de l'État nazi.
    
    \item[Ghetto] Quartier imposé aux Juif$\cdot$ve$\cdot$s par l'État. La population juive est forcée de vivre à l'écart de la population non-juive. Dans l'Allemagne nazie, les ghettos sont clôturés et constamment surveillés. La densité de population y est très élevée et les conditions de vies difficiles. Les autorités y pratiquent l'extermination par la faim et l'épidémie. Avec la \enquote{Solution finale}, les ghettos sont devenus une étape intermédiaire avant la déportation des Juif$\cdot$ve$\cdot$s vers les camps d'extermination.
    
    \item[Lois de Nuremberg] Distinguent la citoyenneté de la nationalité. \enquote{Le Juif} est défini par ses ascendants (au moins trois grands-parents juifs) et ne peut pas être citoyen. Les mariages et relations sexuelles entre Juif$\cdot$ve$\cdot$s et non-Juif$\cdot$ve$\cdot$s sont interdits.
    
    \item[Nazisme] Abréviation de national-socialisme. Idéologie antidémocratique et antimarxiste, nationaliste et pangermaniste.
    
    \item[Nuit de Cristal] (\textit{Kristallnacht}). Suite à l'assassinat d'un conseiller d'ambassade allemand le 7 novembre 1938 à Paris, Joseph Goebbels prononce un discours qui incite la population à une grande violence antisémite. Des \textit{pogroms} ont lieu partout~: 267 synagogues sont pillées, saccagées et incendiées, 7~500 magasins sont dévalisés ou détruits, les habitations des Juif$\cdot$ve$\cdot$s sont saccagées. La \enquote{Nuit de Cristal} dure jusqu'au 10 novembre, dans l'après-midi. Une centaine de Juif$\cdot$ve$\cdot$s sont assassinés, les femmes sont violées malgré l'interdit racial et les enfants juif$\cdot$ve$\cdot$s sont chassé$\cdot$e$\cdot$s des orphelinats. Près de 11~000 hommes sont arrêtés et envoyés à Dachau, et 10~000 à Buchenwald.
    
    \item[Pogrom] Terme russe signifiant 
    \enquote{dévastation} ou \enquote{destruction}. Désigne les violences antisémites commises par la population, incitée par le pouvoir en place. Le régime nazi a provoqué des \textit{pogroms} à de nombreuses reprises.
    
    \item[Question juive] Expression faisant originellement référence à la capacité des Juif$\cdot$ve$\cdot$s à s'intégrer en Europe occidentale. Pour le régime nazi, la \enquote{question juive} est un problème auquel il faut apporter une solution, qui consiste en l'extermination de la population juive.
    
    \item[Shoah] (Holocauste). Terme hébreu signifiant \enquote{catastrophe}, synonyme du terme \enquote{Holocauste} qui, à l'origine, fait référence à un sacrifice dans un but religieux. Le terme fait aujourd'hui référence au génocide des Juif$\cdot$ve$\cdot$s d'Europe par le régime nazi. Les anglophones emploient le terme \enquote{\textit{Holocaust}}, tandis que les francophones parlent de \enquote{Shoah}.
    
    \item[Solution finale] Projet d'extermination des Juif$\cdot$ve$\cdot$s d'Europe par le régime nazi. Ce \enquote{projet} est envisagé comme solution au \enquote{problème} posé par la \enquote{question juive}. La Solution finale accélère le processus entamé par l'enfermement des Juif$\cdot$ve$\cdot$s dans les ghettos et leur déportation dans les camps de concentration. Les premiers camps d'extermination sont créés dès 1941 et pratiquent la mort par le gaz et par balles.
    
    \item[Spoliation] Désigne la confiscation des biens possédés par les Juif$\cdot$ve$\cdot$s. La politique de spoliation orchestrée par le régime nazi a commencé par l'identification de tous les Juif$\cdot$ve$\cdot$s, qui a été suivie par une série de lois visant à les déposséder et anéantir leurs droits.
    
\end{description}