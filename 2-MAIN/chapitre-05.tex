% Définition du nouveau modèle d'encodage

\section{\textit{One Document Does-it-all}}

Une ODD (\enquote{\textit{One Document Does-it-all}}) est un fichier XML encodé en TEI dont l'objectif est double. Elle permet de documenter le projet éditorial, en prose, mais également de créer des règles de validation. Ces dernières seront exportées dans un schéma RelaxNG, qui sera appliqué aux fichiers pour guider et valider leur encodage.

Il existe trois principales façons de générer une ODD~: l'utilisation de l'interface ROMA\footnote{ROMA~: \texttt{\href{https://roma.tei-c.org/}{https://roma.tei-c.org/}}.} créée par le TEI Consortium, l'utilisation de la feuille de style XSL \enquote{\textit{Oddbyexample}} à partir d'un encodage préexistant, ou la rédaction manuelle de l'ODD à partir de rien. Nous avons rapidement éliminé les deux dernières options et choisi de générer la base de notre ODD avec ROMA. En effet, la création de l'ODD manuellement aurait été trop chronophage, et la génération à partir de \enquote{\textit{Oddbyexample}} aurait nécessité de corriger les fichiers EHRI en amont~; or la répartition des balises est parfois inégale au sein des trois cent trente-cinq fichiers.  

La solution la plus optimale dans notre cas a donc été de travailler avec ROMA en partant du schéma le plus large~: \enquote{\textit{TEI All}}. L'application directe de ce schéma serait tout à fait inutile, puisqu'il contient tous les éléments de la TEI, alors que le projet n'en utilise qu'une partie. Il existe d'autres schémas spécifiques mais pour être pertinent, le schéma doit correspondre le plus possible aux caractéristiques du projet.  

La customisation de l'ODD avec ROMA se présente sous la forme d'une liste des éléments de la TEI accompagnés d'une brève présentation et d'une case à cocher. Les éléments peuvent être classés par ordre alphabétique ou par module. Dans leur rapport, les éditeur$\cdot$ice$\cdot$s de l'EHRI ont souligné l'importance du module \texttt{namesdates} dans leur encodage\footcite{Ehri2018}. Pour créer le modèle le plus pertinent possible, nous nous sommes attaché$\cdot$e à consulter la documentation de tous les éléments qui nous paraissaient pertinents, en particulier concernant l'encodage des métadonnées dans le \texttt{<teiHeader>}. Une fois la base de l'ODD générée, il convient de rédiger consciencieusement la documentation et de créer des règles de validation fonctionnelles.  



\section{Rédaction de la documentation}
Nous avons divisé la rédaction de la documentation de l'ODD en trois sections~: les règles fondamentales, l'encodage du \texttt{<teiHeader>} et l'encodage du \texttt{<body>}. L'idée de la documentation de l'ODD est de s'affranchir de la seule consultation des \textit{TEI Guidelines} et d'illustrer l'usage que nous en faisons dans le cadre du projet, notamment grâce à des exemples tirés de notre propre corpus.  

La création du tableau récapitulatif (Annexe \ref{Tableau}), et les commentaires que nous y avons inclus, nous a permis de dégager certaines pratiques que les éditions en ligne de l'EHRI auraient tout intérêt à adopter pour assurer la conformité et l'uniformité de leur encodage. Ce sont les recommandations d'uniformisation que nous évoquions dans le chapitre précédent.  

En analysant l'encodage des collections, nous avons pu adapter la structure du \texttt{<teiHeader>} (Annexe \ref{Metadonnees}). Cette partie de la documentation est particulièrement longue, mais il nous paraissait essentiel d'insister dessus dans la mesure où les membres de l'équipe éditoriale de l'EHRI sont nombreux et que la composition de l'équipe peut être amenée à changer au fil des années. C'est cette partie qui a nécessité le plus de clarifications, dans la mesure où nombre d'éléments figuraient déjà dans l'encodage des métadonnées, mais leur utilisation a été réadaptée dans notre nouveau modèle.  

Enfin, nos recommandations pour l'encodage du \texttt{<body>} ont été guidées par la volonté de transcrire la structure du fac-similé le plus fidèlement possible. Nous avons donc insisté sur l'utilisation des balises \texttt{<pb/>} (\enquote{\textit{page beginning}}) et \texttt{<lb/>} (\enquote{\textit{line beginning}}) pour que l'affichage lors de la publication soit le plus proche possible du fac-similé. Nous avons également affiné l'encodage en introduisant les éléments \texttt{<opener>} et \texttt{<closer>}, contenant tous les deux les informations relatives aux en-têtes des documents et aux éléments de salutations.



\section{Création du schéma de validation}
Les règles du schéma de validation sont contenues dans un élément \texttt{<schemaSpec>} (\enquote{\textit{schema specification}}).  

\subsection{Déclaration des modules}
Les éléments de la TEI sont classés en modules. Quatre d'entre eux sont obligatoires et communs à tous les encodages~: \texttt{tei}, \texttt{header}, \texttt{core} et \texttt{textstructure}. Les modules sont déclarés à l'aide de l'élément \texttt{<moduleRef>} (\enquote{\textit{module reference}}). Le module déclaré est précisé par la valeur de l'attribut \texttt{@key}.  

Il existe ensuite deux façons de déclarer les éléments de chaque module acceptés par le schéma de validation~: l'attribut \texttt{@include} déclare les éléments autorisés, tandis que l'attribut \texttt{@except} déclare les éléments non-autorisés. Dans un souci de légèreté et de lisibilité, il est généralement conseillé de choisir le mode de déclaration en fonction du nombre d'éléments à déclarer. Ainsi, si l'on souhaite utiliser tous les éléments du module \texttt{namesdates} sauf les éléments \texttt{<age>} et \texttt{<birth>}, par exemple, il est conseillé de les exclure avec l'attribut \texttt{@except}. Toutefois, nous avons choisi, dans notre modèle, de n'utiliser que l'attribut \texttt{@include}, que nous estimons plus parlant pour des personnes qui ne seraient pas familières de la TEI.


\subsection{Règles de validation des éléments}
La personnalisation des règles de validation des éléments se fait grâce à l'élément \texttt{<elementSpec>} (\enquote{\textit{element specification}}), qui contient les informations sur la structure et le contenu possible de l'élément. Il n'est pas nécessaire de déclarer ici les éléments qui restent inchangés. Nous avons établi pour les éditions EHRI des règles de validation mettant en place les recommandations exprimées dans la partie documentation. Il s'agissait le plus souvent de déclarer le caractère recommandé (\texttt{"rec"} pour \enquote{\textit{recommended}}) ou obligatoire (\texttt{"req"} pour \enquote{\textit{required}}) d'un attribut, et éventuellement de définir une liste semi-ouverte de valeurs (comme pour l'attribut \texttt{@xml:lang}).


\subsection{Déclaration des classes d'attributs}
Les attributs sont regroupés en classes. Par exemple, la plupart des attributs précisant l'affichage souhaité de l'encodage appartiennent à la classe \texttt{att.global.rendition}. Certains attributs, comme \texttt{@type}, appartiennent à plusieurs classes d'attributs, et chaque élément prend en charge certaines classes définies. La déclaration des classes d'attributs permet donc de limiter l'usage des attributs à celui que l'on souhaite en faire dans le cadre du projet.