% Conclusion

La préparation d'une édition scientifique numérique est un processus long et passionnant. Leur publication permet de créer de formidables outils pour la recherche. Notre expérience nous a confirmé que la définition d'un modèle d'encodage est une étape primordiale afin d'assurer non seulement la validité et la conformité d'une édition, mais également l'uniformité à travers les collections. Le rôle de l'ingénieur$\cdot$e à cet égard est extrêmement important, notamment pour conseiller les chercheur$\cdot$euse$\cdot$s et les aider à naviguer parmi les outils, formats et standards.  

Les éditions de l'EHRI bénéficiant d'un aperçu de deux outils de publication, TEI Publisher et Omeka, les avantages de chacun d'eux ont pu être établis par rapport à un même corpus. Le développement de l'application TEI Publisher pour EHRI suit son cours, et les éditeur$\cdot$ice$\cdot$s pourront bientôt expérimenter l'outil d'annotation intégré à TEI Publisher, qui devrait encore davantage les aider dans leur travail éditorial.